\documentclass[french,twoside,draft]{article}
\usepackage[a4paper,includeheadfoot,top=1.2cm,bottom=1.2cm,left=1.2cm,right=1.2cm]{geometry}
\usepackage[utf8]{luainputenc}%
\usepackage{mathtools}
\usepackage{amssymb}
\usepackage{amsmath}
\usepackage{ifluatex} 
\ifluatex%
\usepackage{fontspec}
\usepackage{atveryend}
\usepackage[normalem]{ulem}

\defaultfontfeatures[EBGaramond]{Ligatures=TeX,Numbers={Lining,Proportional}}
\setmainfont[
    Extension      = .otf,
    UprightFont    = *-Regular,
    ItalicFont     = *-Italic,
    BoldFont       = *-Bold,
    BoldItalicFont = *-BoldItalic
]{EBGaramond}
\usepackage{unicode-math}
% \usepackage{libertine}
\newcommand\lt{<}
\newcommand\gt{>}
\else%
  \usepackage[utf8]{inputenc}%
\fi
\usepackage{xcolor}
\usepackage{dsfont}
\usepackage{suffix}
\usepackage{soul}
\newcommand\iddots{\reflectbox{$\ddots$}}
\usepackage{expl3}
\usepackage{etoolbox}
\usepackage{xstring}
\usepackage{xparse}

\newcommand{\onlyprogcolle}[1]{}
\newcommand{\altprogcolle}[2]{#1}
\newcommand{\onlyprogcollephantom}[1]{\phantom{#1}}

\ExplSyntaxOn
\def\diffd{\mathrm{d}}
\ExplSyntaxOff

\usepackage[french]{babel}
\frenchbsetup{og = «, fg = »}
\usepackage[activate={true,nocompatibility},final,tracking=true,babel=true]{microtype}

% \makeatletter
% \def\stripzero#1{\expandafter\strip@zero@help#1}
% \def\strip@zero@help#1{\ifx 0#1\expandafter\strip@zero@help\else#1\fi}
% \makeatother
\author{Bosché Aurélien}
\def\contractedauthorname{A.~Bosché}
\makeatletter
\newcommand*{\rom}[1]{\expandafter\@slowromancap\romannumeral #1@}
\makeatother
\newcommand\etablissement{Lycée militaire d'Autun}

\usepackage{ellipsis}
\definecolor{ocre}{RGB}{243,102,25}
\definecolor{teal}{RGB}{243,102,25}
\definecolor{lightgreen}{RGB}{50,200,50}
\usepackage{datenumber}
\setstartyear{2000}
\usepackage{titlecaps}

\newcommand{\vcenteredinclude}[2][]{\begingroup
\setbox0=\hbox{\includegraphics[#1]{#2}}%
\parbox{\wd0}{\box0}\endgroup}

\usepackage{siunitx}
\sisetup{locale = FR, binary-units=true}
\DeclareSIUnit \shortbit  { b }
\usepackage{pdftexcmds}
\usepackage{fancyhdr}
\usepackage{textcomp}

\usepackage[explicit]{titlesec}
\definecolor{titlebgdark}{RGB}{255,255,255}
\definecolor{titlebglight}{RGB}{191,233,251}
\definecolor{othertitlebgdark}{RGB}{0,243,163}
\definecolor{othertitlebglight}{RGB}{191,251,233}
\definecolor{CoverLeftBarColor}{RGB}{191,233,251}
\definecolor{CoverUpperBarColor}{RGB}{0,163,243}

\usepackage{mathrsfs}
\usepackage{stmaryrd}
\usepackage{esvect}
% \usepackage{dsfont}
% \DeclareRobustCommand{\officialeuro}{€}
% % \ifmmode\expandafter\text\fi%
% % {\fontencoding{U}\fontfamily{eurosym}\selectfont e}}

\usepackage[np]{numprint}
\usepackage{icomma}
\usepackage[newcommands]{ragged2e}

\usepackage{multicol}
\usepackage{booktabs}

\newcommand*{\myll}[1]{\underset{#1}{\ll}}

\usepackage{enumitem}
\setlist[itemize,1]{label={\textbullet}}

\makeatletter{}
\enitkv@key{}{nbcols}{%
  \ifnumcomp{#1}{=}{1}{}{%
    \def\enit@before{%
      \setlength{\multicolsep}{\abovedisplayskip}
      \begin{multicols}{#1}}%
      \def\enit@after{\vspace*{\fill}\end{multicols}}%
  }%
}
\makeatother{}


\usepackage{pageslts}

\usepackage{tikz}
\usetikzlibrary{matrix,calc,arrows,arrows.meta,positioning,tikzmark,decorations.pathreplacing,decorations.markings,shapes}
\tikzset{>=latex}
\newcommand*\keystroke[1]{%
  \tikz[baseline=(key.base)]%
    \node[%
      draw,%
      fill=white,%
      rectangle,%
      rounded corners=2pt,%
      inner sep=1pt,%
      line width=0.5pt,%
      font=\scriptsize\sffamily%
    ](key) {#1\strut};%
}
\usepackage{tkz-tab}
\tikzset{t style/.style = {style = solid}}
\presetkeys     [TAB] {tbs} {nocadre  = true}{}

\usepackage[skins,theorems,breakable,xparse]{tcolorbox}
\tcbset{
  label separator=:,
  geometry nodes=true,
  enhanced
}

\newcommand{\scsep}{\,;\,}
\newcommand{\comsep}{\,,\,}

\newcommand\tuple[1]{%
  \begingroup
  \def\nextitem{\def\nextitem{\comsep}}%
  \renewcommand*{\do}[1]{\nextitem{}##1}%
  \left(\docsvlist{#1}\right)%
  \endgroup
}
\newcommand\set[1]{%
  \begingroup
  \def\nextitem{\def\nextitem{\comsep}}%
  \renewcommand*{\do}[1]{\nextitem{}##1}%
  \left\{\docsvlist{#1}\right\}%
  \endgroup
}
\robustify\tuple
\robustify\set

\newcounter{tmp}
\newcommand\permut[1]{%
  \begingroup
  \setcounter{tmp}{0}
  \forcsvlist{\listadd\mylist}{#1}%
  \gdef\nextitemone{\gdef\nextitemone{&}}%
  \gdef\nextitemtwo{\gdef\nextitemtwo{&}}%
  \newcommand\glob[1]{}
  \begin{pmatrix}
    \forlistloop{%
      \nextitemtwo\stepcounter{tmp}\arabic{tmp}\glob}{\mylist} \\%
    \forlistloop{\nextitemone{}}{\mylist}
  \end{pmatrix}
  \endgroup
}

\newcommand\permutbyhand[2]{%
  \begingroup
  \setcounter{tmp}{0}
  \forcsvlist{\listadd\mylistone}{#1}%
  \forcsvlist{\listadd\mylisttwo}{#2}%
  \gdef\nextitemone{\gdef\nextitemone{&}}%
  \gdef\nextitemtwo{\gdef\nextitemtwo{&}}%
  \begin{pmatrix}
    \forlistloop{\nextitemone{}}{\mylistone} \\
    \forlistloop{\nextitemtwo{}}{\mylisttwo}
  \end{pmatrix}
  \endgroup
}

\newcommand\perpsum[1]{\overset{\perp}{\oplus}}
\newcommand\dirsum[1]{\oplus}
\newcommand\setinterior[1]{\mathring{#1}}
\DeclareSymbolFont{yhlargesymbols}{OMX}{yhex}{m}{n}
\DeclareMathAccent{\wideparen}{\mathord}{yhlargesymbols}{"F3}
\newcommand\setinteriorbig[1]{\mathring{\wideparen{#1}}}
\newcommand\setclosure[1]{\overline{#1}}
\usepackage{adjustbox}
\newcommand\subline[2]{\overline{#1}^{\,\stackbox[t][l]{#2}}}
\newcommand\setclosurein[2]{\subline{#1}{#2}}
\newcommand\setclosureinRbar[1]{\setclosurein{#1}{\scriptsize\Rbar}}

\DeclareListParser{\doscsvlist}{;}

\newcommand\vvect[2]{%
  \delimitershortfall=0pt
  \delimiterfactor=1000
  %
  \begingroup
  \def\nextitem{\def\nextitem{\\}}%
  \renewcommand*{\do}[1]{\nextitem{}##1}%
  \ifx\relax#1\relax%
  \else%
  \vv{#1}%
  \fi%
  \left(%
    \begin{smallmatrix}\doscsvlist{#2}\end{smallmatrix}%
  \right)%
  \endgroup
}
\newcommand\vvectnoname[1]{%
  \delimitershortfall=0pt
  \delimiterfactor=1000
  %
  \begingroup
  \def\nextitem{\def\nextitem{\\}}%
  \renewcommand*{\do}[1]{\nextitem{}##1}%
  \left(%
    \begin{smallmatrix}\doscsvlist{#1}\end{smallmatrix}%
  \right)%
  \endgroup
}


\newcommand\setcompr[2]{%
  \left\{#1\mid{}~{}#2\right\}
}
\newcommand\setext[2]{%
  \left\{#1;~{}#2\right\}
}

\newcommand\signature[1]{\epsilon\left(#1\right)}

% Interdire les coupures dans les formules de maths.
% \relpenalty=10000%
% \binoppenalty=10000%

\usepackage{pgfplots}
\usepgfplotslibrary{colormaps}
\pgfplotsset{%
  compat=newest,%
  legend pos=north west,%
  samples=100,%
  /pgf/number format/use comma,%
  major tick length=0.2cm,%
  minor tick length=0.1cm,%
  every major tick/.append style={black},%
  every axis x label/.style={at={(current axis.right of origin)},%
    anchor=south east},%
  every axis y label/.style={at={(current axis.above origin)},%
    anchor=north west},%
  every axis/.append style={ultra thick},%
  every major grid/.append style={gray!80!black},%
  grid=both,%
  tick label style={fill=white},%
  axis lines=center,%
  minor x tick num=1,%
  /pgf/number format/use comma,%
  major tick length=0.2cm,%
  minor tick length=0.1cm,%
  every major tick/.append style={black, very thick},%
  every minor tick/.append style={black, very thick},%
  every axis x label/.style={at={(current axis.right of origin)},%
    anchor=south east},%
  every axis y label/.style={at={(current axis.above origin)},%
    anchor=north west},%
  every axis/.append style={ultra thick},%
  enlarge x limits=0.05,%
  enlarge y limits=0.05,%
}
  \pgfplotsset{
    colormap={lightgray}{rgb=(0.3,0.3,0.3)
      rgb=(0.9,0.9,0.9)
    }
  }

\usepackage{tabularx}
\definecolor{graycell}{gray}{0.75}
\definecolor{graycol}{gray}{0.75}
\def\tabularxcolumn#1{m{#1}}
\newcolumntype{M}{>{\Centering$}X<{$}}
\newcolumntype{L}{>{\RaggedRight}X<{}}
\newcolumntype{R}{>{\RaggedLeft}X<{}}
\newcolumntype{C}{>{\Centering}X<{}}
\newcolumntype{G}{>{\columncolor{graycol}}c<{}}

%\DeclareMathOperator{\deg}{deg}
\DeclareMathOperator{\Arg}{Arg}
\DeclareMathOperator{\len}{len}
\let\arg\relax
\DeclareMathOperator{\arg}{arg}
\DeclareMathOperator{\sgn}{sgn}
\DeclareMathOperator{\cotan}{cotan}
\DeclareMathOperator{\arccotan}{arccotan}

\DeclareMathOperator{\Ker}{Ker}
\DeclareMathOperator{\im}{Im}
\DeclareMathOperator{\coker}{Coker}
\DeclareMathOperator{\Mat}{Mat}
\DeclareMathOperator{\cov}{Cov}

\DeclareMathOperator{\Vect}{Vect}
\DeclareMathOperator{\rg}{rg}
\DeclareMathOperator{\tr}{Tr}
\DeclareMathOperator{\diag}{diag}
\DeclareMathOperator{\Sp}{Sp}
\DeclareMathOperator{\spectre}{Sp}
\DeclareMathOperator{\cardt}{Card}
\let\card\cardt
\DeclareMathOperator{\id}{id}
\DeclareMathOperator{\grad}{\vec{\rm grad}}
\DeclareMathOperator{\Hom}{Hom}
\DeclareMathOperator{\pgcdop}{pgcd}
\DeclareMathOperator{\ppcmop}{ppcm}

\let\ln\relax
\let\th\relax
\DeclareMathOperator{\ln}{ln}
\DeclareMathOperator{\ch}{ch}
\DeclareMathOperator{\sh}{sh}
% \DeclareMathOperator{\card}{card}
\DeclareMathOperator{\comat}{Com}
\DeclareMathOperator{\imv}{Im}
\DeclareMathOperator{\rang}{rg}
\DeclareMathOperator{\Fr}{Fr}
\DeclareMathOperator{\diam}{diam}
\DeclareMathOperator{\supp}{supp}
% \DeclareMathOperator{\inf}{inf}

\newcommand\pgcd[2]{{#1}\wedge{}{#2}}
\newcommand\ppcm[2]{{#1}\vee{#2}}
\newcommand\inbase[2]{\left(#2\right)_{#1}}
\newcommand\insbase[2]{\left(#2\right)^{s}_{#1}}
\newcommand\inubase[2]{\left(#2\right)^{u}_{#1}}

\newcommand\Ox{(Ox)}
\newcommand\Oy{(Oy)}
\newcommand\Oz{(Oy)}

\usepackage{stackrel}
\newcommand{\normn}[2]{\norm{#2}_{#1}}

\newcommand{\reprgraph}[1]{\mathcal{C}_{#1}}
\newcommand{\loceq}[1]{\stackrel[#1]{}{=}}
\newcommand{\locsim}[1]{\stackrel[#1]{}{\sim}}
\newcommand{\BO}[2][]{%
  \ifx\relax#1\relax%
  \operatorname{O}\left(#2\right)%
  \else%
  \operatorname{O}_{#1}\left(#2\right)%
  \fi%
}
\newcommand{\lo}[2][]{%
  \ifx\relax#1\relax%
  \operatorname{o}\left(#2\right)%
  \else%
  \operatorname{o}_{#1}\left(#2\right)%
  \fi%
}

\newcommand{\locBO}[2]{\BO[#1]{#2}}%
\newcommand{\loclo}[2]{\lo[#1]{#2}}%

\robustify\arg
\robustify\Arg
\robustify\lo
\robustify\BO
\robustify\locsim
\robustify\locBO
\robustify\loclo
\robustify\operatorname
\robustify\implies
\robustify\iff
\robustify\int
\robustify\sum
\robustify\prod

\newcommand{\BT}[1]{\operatorname{\Theta}\left(#1\right)}
\newcommand\planC{\mathcal{P}}


\newcommand\repere[3]{\left(#1;#2,#3\right)}
\newcommand\OIJ{\repere{O}{I}{J}}

\newcommand\K{\mathbf{K}}
\newcommand\quaternionsset{\mathbf{H}}
\let\C\undefined
\newcommand\C{\mathbf{C}}
\newcommand\R{\mathbf{R}}
\newcommand\iR{i\mathbf{R}}
\newcommand\Q{\mathbf{Q}}
\newcommand\Z{\mathbf{Z}}
\newcommand\N{\mathbf{N}}
\newcommand\DD{\mathbf{D}}
\newcommand\Kst{\mathbf{K}^{\ast}}
\newcommand\Cst{\mathbf{C}^{\ast}}
\newcommand\Rst{\mathbf{R}^{\ast}}
\newcommand\Qst{\mathbf{Q}^{\ast}}
\newcommand\Zst{\mathbf{Z}^{\ast}}
\newcommand\Nst{\mathbf{N}^{\ast}}
\newcommand\DDst{\mathbf{D}^{\ast}}
\newcommand\setst[1]{#1^{\ast}}
\newcommand\Kp{\mathbf{K}_{+}}
\newcommand\Rp{\mathbf{R}_{+}}
\newcommand\Qp{\mathbf{Q}_{+}}
\newcommand\DDp{\mathbf{D}_+}
\newcommand\Zp{\mathbf{Z}_{+}}
\newcommand\Np{\mathbf{N}_{+}}
\newcommand\Km{\mathbf{K}_{-}}
\newcommand\Cm{\mathbf{C}_{-}}
\newcommand\Rm{\mathbf{R}_{-}}
\newcommand\Qm{\mathbf{Q}_{-}}
\newcommand\Zm{\mathbf{Z}_{-}}
\newcommand\Nm{\mathbf{N}_{-}}
\newcommand\zp{0^{+}}
\newcommand\zm{0^{-}}
\newcommand\Knamecr[1]{\mathbf{#1}[X]}
\newcommand\Knamecrn[2]{\mathbf{#1}_{#2}[X]}
\newcommand\Knamepar[1]{\mathbf{#1}(X)}
\newcommand\Kcrn[1]{\mathbf{K}_{#1}[X]}
\newcommand\Qcrn[1]{\mathbf{Q}_{#1}[X]}
\newcommand\Ccrn[1]{\mathbf{C}_{#1}[X]}
\newcommand\Rcrn[1]{\mathbf{R}_{#1}[X]}
\newcommand\Zcrn[1]{\mathbf{Z}_{#1}[X]}
\newcommand\Kcr{\mathbf{K}[X]}
\newcommand\Qcr{\mathbf{Q}[X]}
\newcommand\Ccr{\mathbf{C}[X]}
\newcommand\Rcr{\mathbf{R}[X]}
\newcommand\Zcr{\mathbf{Z}[X]}
\newcommand\Kpar{\mathbf{K}(X)}
\newcommand\Qpar{\mathbf{Q}(X)}
\newcommand\Cpar{\mathbf{C}(X)}
\newcommand\Rpar{\mathbf{R}(X)}
\newcommand\Kcrnst[1]{\mathbf{K}_{#1}[X]^{\ast}}
\newcommand\Ccrnst[1]{\mathbf{C}_{#1}[X]^{\ast}}
\newcommand\Rcrnst[1]{\mathbf{R}_{#1}[X]^{\ast}}
\newcommand\Kcrst{\mathbf{K}[X]^{\ast}}
\newcommand\Ccrst{\mathbf{C}[X]^{\ast}}
\newcommand\Rcrst{\mathbf{R}[X]^{\ast}}
\newcommand\commutant[1]{\mathcal{C}\left(#1\right)}

\newcommand\MmnK[2]{\mathcal{M}_{#1,#2}\left(\K\right)}
\newcommand\MnK[1]{\mathcal{M}_{#1}\left(\K\right)}
\newcommand\MmnR[2]{\mathcal{M}_{#1,#2}\left(\R\right)}
\newcommand\MnR[1]{\mathcal{M}_{#1}\left(\R\right)}
\newcommand\MmnC[2]{\mathcal{M}_{#1,#2}\left(\C\right)}
\newcommand\MnC[1]{\mathcal{M}_{#1}\left(\C\right)}
\newcommand\MmnQ[2]{\mathcal{M}_{#1,#2}\left(\Q\right)}
\newcommand\MnQ[1]{\mathcal{M}_{#1}\left(\Q\right)}
\newcommand\MatPass[2]{P_{#1}^{#2}}
\newcommand\MatFamille[2]{P_{#1}\left(#2\right)}
\newcommand\Colonne[2]{C_{#1}\left(#2\right)}
\newcommand\Ligne[2]{L_{#1}\left(#2\right)}
%
%
\newcommand\evcont[1]{\overline{#1}}
\newcommand\pcond[2]{p_{#2}\left(#1\right)}
\newcommand\pcondmid[2]{p\left(#1\mid#2\right)}
%
\newcommand\pscal[2]{\left(#1\mid#2\right)}
\newcommand\pscalang[2]{\langle#1\,,\,#2\rangle}
\newcommand\orth[1]{#1^{\perp}}
\newcommand\dist[2]{d\left(#1,#2\right)}
% \newcommand\gporth[1]{O\left(#1\right)}
\newcommand\OEE[1]{\text{O}\left(#1\right)}
\newcommand\OnR[1]{\text{O}_{#1}\left(\R\right)}
\newcommand\SOnR[1]{\text{SO}_{#1}\left(\R\right)}
\newcommand\SLnR[1]{\text{SL}_{#1}\left(\R\right)}
\newcommand\SOE[1]{\text{SO}\left(#1\right)}
\newcommand\SLE[1]{\text{SL}\left(#1\right)}
% \newcommand{\revdots}{\mathinner{\mkern1mu\raise1pt\vbox{\kern7pt\hbox
% {.}}\mkern2mu\raise4pt\hbox{.}\mkern2mu\raise7pt\hbox
% {.}\mkern1mu}}

\newcommand\TnsK[1]{\mathcal{T}^{s}_{#1}\left(\K\right)}
\newcommand\TnsR[1]{\mathcal{T}^{s}_{#1}\left(\R\right)}
\newcommand\TnsC[1]{\mathcal{T}^{s}_{#1}\left(\C\right)}
\newcommand\TniK[1]{\mathcal{T}^{i}_{#1}\left(\K\right)}
\newcommand\TniR[1]{\mathcal{T}^{i}_{#1}\left(\R\right)}
\newcommand\TniC[1]{\mathcal{T}^{i}_{#1}\left(\C\right)}
\newcommand\SnK[1]{\mathcal{S}_{#1}\left(\K\right)}
\newcommand\SnR[1]{\mathcal{S}_{#1}\left(\R\right)}
\newcommand\SnC[1]{\mathcal{S}_{#1}\left(\C\right)}
\newcommand\AnK[1]{\mathcal{A}_{#1}\left(\K\right)}
\newcommand\AnR[1]{\mathcal{A}_{#1}\left(\R\right)}
\newcommand\AnC[1]{\mathcal{A}_{#1}\left(\C\right)}

\newcommand\rootmult[2]{m_{#1}(#2)}
\newcommand\rootset[2]{\mathcal{R}_{#1}(#2)}


\newcommand\poltaylor[3]{T_{#1,#2,#3}}
\newcommand\restetaylor[3]{R_{#1,#2,#3}}

\robustify\K
\robustify\R
\robustify\iR
\robustify\Q
\robustify\Z
\robustify\N
\robustify\DD
\robustify\DDst
\robustify\Kst
\robustify\C
\robustify\Cst
\robustify\Rst
\robustify\Qst
\robustify\Zst
\robustify\Nst
\robustify\Kp
\robustify\Rp
\robustify\Qp
\robustify\Zp
\robustify\Np
\robustify\Km
\robustify\Cm
\robustify\Rm
\robustify\Qm
\robustify\Zm
\robustify\Nm
\robustify\ch
\robustify\sh
\robustify\arcsin
\robustify\arccos
\robustify\arctan
\robustify\sqrt
\robustify\exp
\robustify\substack
\robustify\mod
\robustify\vv

\robustify\Kcr
\robustify\Rcr
\robustify\Ccr
\robustify\Qcr
\robustify\Kcrn
\robustify\Rcrn
\robustify\Ccrn
\robustify\Qcrn
\robustify\rootset

\newcommand\setp[1]{#1_{+}}
\newcommand\setm[1]{#1_{-}}

\newcommand\Kpst{\mathbf{K}^{\ast}_{+}}
\newcommand\Cpst{\mathbf{C}^{\ast}_{+}}
\newcommand\Rpst{\mathbf{R}^{\ast}_{+}}
\newcommand\Qpst{\mathbf{Q}^{\ast}_{+}}
\newcommand\Zpst{\mathbf{Z}^{\ast}_{+}}
\newcommand\Npst{\mathbf{N}^{\ast}_{+}}

\robustify\Kpst
\robustify\Cpst
\robustify\Rpst
\robustify\Qpst
\robustify\Zpst
\robustify\Npst

\newcommand\Kmst{\mathbf{K}^{\ast}_{-}}
\newcommand\Rmst{\mathbf{R}^{\ast}_{-}}
\newcommand\Qmst{\mathbf{Q}^{\ast}_{-}}
\newcommand\Zmst{\mathbf{Z}^{\ast}_{-}}
\newcommand\Nmst{\mathbf{N}^{\ast}_{-}}

\robustify\Kmst
\robustify\Rmst
\robustify\Qmst
\robustify\Zmst
\robustify\Nmst

\newcommand\setpst[1]{#1^{\ast}_{+}}
\newcommand\setmst[1]{#1^{\ast}_{-}}


\newcommand\nthrootgroup[1]{\mathbf{U}_{#1}}
\newcommand\unitgroup{\mathbf{U}}
\robustify\unitgroup
\robustify\nthrootgroup

\newcommand*\oline[1]{%
   \vbox{%
     \hrule height 0.5pt%                  % Line above with certain width
     \kern0.25ex%                          % Distance between line and content
     \hbox{%
       \kern-0.1em%                        % Distance between content and left side of box, negative values for lines shorter than content
       \ifmmode#1\else\ensuremath{#1}\fi%  % The content, typeset in dependence of mode
       \kern-0.1em%                        % Distance between content and left side of box, negative values for lines shorter than content
     }% end of hbox
   }% end of vbox
}
\newcommand\Rbar{\oline{\mathbf{R}}}
\newcommand\Cbar{\oline{\mathbf{C}}}


\def\Ibar{\setclosure{I}}%
\def\IbarinRbar{\setclosureinRbar{I}}%
\def\Irond{\setinterior{I}}%
\def\Ebar{\setclosure{E}}%
\def\EbarinRbar{\setclosureinRbar{E}}%
\def\Erond{\setinterior{E}}%
\def\Jbar{\setclosure{J}}%
\def\JbarinRbar{\setclosureinRbar{J}}%

\robustify\Ibar
\robustify\IbarinRbar
\robustify\Irond
\robustify\Ebar
\robustify\EbarinRbar
\robustify\Erond
\robustify\Jbar
\robustify\JbarinRbar
\robustify\Rbar
\robustify\Cbar

\newcommand\class[1]{\mathcal{C}^{#1}}
\newcommand\kronecker[1]{\delta_{#1}}
\newcommand\setdiffsym[2]{#1\mathbin{\Delta}#2}
\newcommand\setdiff[2]{#1∖#2}
\newcommand\setcompvar[2]{{\normalsize\complement}^{#2}_{#1}}
\newcommand\setcomp[2]{\setdiff{#1}{#2}}
\newcommand\setcompbar[1]{\overline{#1}}
\newcommand\powerset[1]{\mathscr{P}(#1)}
\newcommand\ppowerset[2]{\mathscr{P}_{#1}(#2)}
\newcommand\applisetpower[2]{#2^{#1}}
\newcommand\applisetinline[2]{\mathcal{F}(#1,#2)}
\newcommand\appliset[2]{\applisetpower{#1}{#2}}
\newcommand\permset[1]{\mathbf{S}_{#1}}
\newcommand\permsetn[1]{\text{S}_{#1}}
\newcommand\permsetfrakn[1]{\frak{S}_{#1}}
\newcommand\cnset[3]{\mathcal{C}^{#1}\left(#2,#3\right)}
\newcommand\contM[2]{\mathcal{C}_{\mathcal{M}}\left(#1,#2\K\right)}
\newcommand\contMR[1]{\mathcal{C}_{\mathcal{M}}\left(#1,\R\right)}
\newcommand\contMK[1]{\mathcal{C}_{\mathcal{M}}\left(#1,\K\right)}
\newcommand\escasetK[1]{\mathcal{E}\left(#1,\K\right)}
\newcommand\Sol{\mathcal{S}}
\newcommand\invset[1]{\text{U}\left(#1\right)}

\robustify\powerset
\robustify\applisetpower
\robustify\applisetinline
\robustify\appliset

\delimitershortfall=-1.2pt
\delimiterfactor=1100

\newcommand\x{\times}
\newcommand\lp{\left\{}
\newcommand\rp{\right\}}

\newcommand\reste[2]{#1\mathbin{\%}#2}
\newcommand\quotient[2]{#1\mathbin{//}#2}

\usepackage{centernot}
\newcommand\relbinR{\mathrel{R}}
\newcommand\relbin{\mathrel{R}}
\newcommand\nrelbin{\mathrel{\rlap{\,/}R}}
\newcommand\primeset{\mathcal{P}}
\newcommand\irrset[1]{\mathcal{I}_{\mathbf{#1}[X]}}
\newcommand\irrsetunit[1]{\mathcal{I}^{u}_{\mathbf{#1}[X]}}
\newcommand\valpadic[2]{v_{#1}\left(#2\right)}
\newcommand\classequi[1]{\operatorname{Cl}\left(#1\right)}
\newcommand\relequi{\sim{}}
\newcommand\dblrestr[3]{#1_{\left|#2\right.}^{\left|#3\right.}}
\newcommand\corestr[2]{#1^{\left|#2\right.}}
\newcommand\restr[2]{#1_{\left|#2\right.}}
\newcommand\imdir[2]{{#1}^{\rightarrow}{\left(#2\right)}}
\newcommand\imrecip[2]{{#1}^{\leftarrow}{\left(#2\right)}}
\newcommand\invimg[2]{#1^{-1}(#2)}
\newcommand\dirimg[2]{#1(#2)}
\newcommand{\bod}[1]{\allowbreak
  \if@display\mkern18mu\else\mkern8mu\fi\left[\,#1\,\right]}

\robustify\cdots
\robustify\irrset
\robustify\irrsetunit
\robustify\reste
\robustify\quotient
\robustify\relbinR
\robustify\relbin
\robustify\nrelbin
\robustify\primeset
\robustify\valpadic
\robustify\classequi
\robustify\relequi
\robustify\dblrestr
\robustify\corestr
\robustify\restr
\robustify\invimg
\robustify\dirimg
\robustify\bod



\newcommand\suitloibern[2]{#1\hookrightarrow\mathcal{B}(#2)}
\newcommand\loibern[1]{\mathcal{B}(#1)}
\newcommand\suitloibinom[3]{#1\hookrightarrow\mathcal{B}(#2,#3)}
\newcommand\loibinom[2]{\mathcal{B}(#1,#2)}
\newcommand\suitloiunif[2]{#1\hookrightarrow\mathcal{U}(#2)}
\newcommand\loiunif[1]{\mathcal{U}(#1)}
\newcommand\esp[1]{\text{E}\left(#1\right)}
\newcommand\variance[1]{\text{V}\left(#1\right)}
\newcommand\covariance[2]{\text{cov}\left(#1,#2\right)}
\newcommand\moment[2]{m_{#1}\left(#2\right)}
\newcommand\ectype[1]{\sigma\left(#1\right)}

\renewcommand\vec[1]{\vv{#1}}
\newcommand\opligne[1]{L_{#1}}
\newcommand\opcol[1]{C_{#1}}
\newcommand\opcolmat[2]{C_{#1}\left(#2\right)}
\newcommand\oplex[2]{L_{#1}\leftrightarrow{}L_{#2}}
\newcommand\opcex[2]{C_{#1}\leftrightarrow{}C_{#2}}
\newcommand\opltransv[4][+]{L_{#2}\leftarrow{}L_{#2}{}#1{}#3{}L_{#4}}
\newcommand\opctransv[4][+]{C_{#2}\leftarrow{}C_{#2}{}#1{}#3{}C_{#4}}
\newcommand\opltransvm[3]{L_{#1}\leftarrow{}L_{#1}{}-{}#2{}L_{#3}}
\newcommand\opctransvm[3]{C_{#1}\leftarrow{}C_{#1}{}-{}#2{}C_{#3}}
\newcommand\opltransvp[3]{L_{#1}\leftarrow{}L_{#1}{}+{}#2{}L_{#3}}
\newcommand\opctransvp[3]{C_{#1}\leftarrow{}C_{#1}{}+{}#2{}C_{#3}}
\newcommand\opldilat[2]{L_{#1}\leftarrow{}#2{}L_{#1}}
\newcommand\opcdilat[2]{C_{#1}\leftarrow{}#2{}C_{#1}}

\newcommand\matex[2]{P_{#1,#2}}
\newcommand\mattransv[3]{T_{#1,#3}\left(#2\right)}
\newcommand\matdil[2]{D_{#1}\left(#2\right)}
% \newcommand\matextaille[3]{P_{#1,#2}^{#3}}
% \newcommand\mattransvtaille[4]{T_{#1,#3}^{#4}\left(#3\right)}
% \newcommand\matdiltaille[3]{D_{#1}^{#3}\left(#2\right)}

\newcommand\fconst[1]{\tilde{#1}}
\newcommand\mytilde[1]{\tilde{#1}}
\robustify\mytilde
\newcommand\indic[1]{\mathbf{1}_{#1}}
\newcommand\ensparties[1]{\mathbf{1}_{#1}}

\newcommand{\veci}{\vec{\imath}}
\newcommand{\vecj}{\vec{\jmath}}
\newcommand{\veck}{\vec{k}}

\makeatletter
\newcommand\vcolwithname[3][\\]{%
    \global\def\my@delim{#1}%
    #2\,\begin{pmatrix}
        \my@vector #3,\relax\noexpand\@eolst%
      \end{pmatrix}}
\newcommand\cycle[2][~]{%
      \global\def\my@delim{#1}%
      \begin{pmatrix}
        \my@vector #2,\relax\noexpand\@eolst%
    \end{pmatrix}}
\newcommand\vcol[2][\\]{%
    \global\def\my@delim{#1}%
    \begin{pmatrix}
        \my@vector #2,\relax\noexpand\@eolst%
    \end{pmatrix}}
\def\my@vector #1,#2\@eolst{%
   \ifx\relax#2\relax
      #1
   \else
      #1\my@delim
      \my@vector #2\@eolst%
      \fi}
\makeatother
    
% Extension to amsmath matrices
\makeatletter
\renewcommand*\env@matrix[1][*\c@MaxMatrixCols c]{%
  \hskip -\arraycolsep
  \let\@ifnextchar\new@ifnextchar
  \array{#1}}
\makeatother

\newcommand\dd{\,\mathrm{d}}
\newcommand\dx{\dd{}x}
\newcommand\dy{\dd{}y}
\newcommand\dt{\dd{}t}
\newcommand\ds{\dd{}s}
\newcommand\du{\dd{}u}
\newcommand\dv{\dd{}v}
\newcommand\dw{\dd{}w}

\DeclarePairedDelimiter\cardv{\lvert}{\rvert}%
\DeclarePairedDelimiter\abs{\lvert}{\rvert}%
\DeclarePairedDelimiter\crochetmod{[}{]}%
\newcommand\cmod[1]{\,\crochetmod{#1}}
\DeclarePairedDelimiter\norm{\lVert}{\rVert}%
\DeclarePairedDelimiter\ceil{\lceil}{\rceil}
\DeclarePairedDelimiter\floor{\lfloor}{\rfloor}
\DeclarePairedDelimiter\vari{[}{]}
\DeclarePairedDelimiter\prodscasurr{<}{>}
\newcommand\prodsca[2]{\prodscasurr{#1\comsep{}#2}}

\newcommand\ppos[1]{#1^{+}}
\newcommand\pneg[1]{#1^{-}}

\makeatletter
\let\oldabs\abs
\def\abs{\@ifstar{\oldabs}{\oldabs*}}
%
\let\oldfloor\floor
\def\floor{\@ifstar{\oldfloor}{\oldfloor*}}
%
\let\oldvari\vari
\def\vari{\@ifstar{\oldvari}{\oldvari*}}
%
\robustify\dd
\robustify\dx
\robustify\dy
\robustify\dt
\robustify\ds
\robustify\du
\robustify\dv
\robustify\dw
\robustify\abs
\robustify\norm
\robustify\dist
\robustify\floor
\robustify\vari
\makeatother

\usepackage{marvosym}
\usepackage{float}
\newfloat{algorithm}{h}{alg}
\floatname{algorithm}{Algorithme}

\usepackage{caption}
\captionsetup{margin=0.05\linewidth}

\usepackage[bookmarks=true]{hyperref}
\usepackage{bookmark}
\usepackage{cleveref}
\newcommand\numberthis{\addtocounter{equation}{1}\tag{\theequation}}

\newcommand\identitymatrix[1][]{\mathbbm{1}_{#1}}
\newcommand\zeromatrix[1][]{\mathbbm{0}_{#1}}

\newcommand\KDelta[1]{\KroneckerDelta{#1}}
\newcommand\transposeRight[1]{#1^{T}}
\newcommand\transposeLeft[1]{\prescript{t}{}{#1}}
\newcommand\transpose[1]{\transposeRight{#1}}

\newcommand{\fullfunction}[6][t]{
  \begin {array}[#1]{rcl}
    {#2} \colon {#3} & \to & {#4} \\
    {#5} & \mapsto     & {#6}
  \end{array}
}
\newcommand{\fullfunctioninline}[5]{%
    {#1}:{#2}\to{#3};~{#4}\mapsto{#5}%
}
\newcommand{\inlinefunction}[3]{
  {#1}\colon{#2}{}\to{}{#3}
}
\newcommand{\inlinemapsto}[2]{
  {#1}\mapsto{#2}%
}

\newcommand{\inlinemapstonamed}[3]{
  \ifx\relax#1\relax%
  {#2}\mapsto{#3}%
  \else%
  {#1}\colon{#2}\mapsto{#3}%
  \fi%
}

\newcommand{\inlinelimit}[5][t]{
  \if#1t\relax%
  #2\xrightarrow{#3\rightarrow{}#4}#5
  \else%
  \if#1b\relax%
  #2\xrightarrow[#3\rightarrow{}#4]{}#5
  \else%
  \if#1m\relax%
  #2\xrightarrow[#4]{#3}#5
  \else%
  ??????????????????
  \fi%
  \fi%
  \fi%
}

\newcommand{\inlineseqlimit}[3][n\to{}+\infty]{
  #2\xrightarrow{#1}#3
}

\newcommand{\inlinecondlimit}[5]{
  #1\xrightarrow[#4]{#2\rightarrow{}#3}#5
}

\newcommand{\inlineleftlimit}[4]{
  \inlinecondlimit{#1}{#2}{#3}{#2<#3}{#4}
}

\newcommand{\inlinerightlimit}[4]{
  \inlinecondlimit{#1}{#2}{#3}{#2>#3}{#4}
}

\newcommand{\inlinedifflimit}[4]{
  \inlinecondlimit{#1}{#2}{#3}{#2\neq{}#3}{#4}
}

\newcommand{\image}[2]{
  #1\left(#2\right)
}
\newcommand{\imager}[2]{
  #1^{-1}\left(#2\right)
}

\newcommand{\dlna}[2]{
  DL_{#1}\left(#2\right)
}

\newcommand\lineappset[2]{\mathcal{L}\left(#1,#2\right)}
\newcommand\lineendo[1]{\mathcal{L}\left(#1\right)}
\newcommand\gl[1]{\text{GL}\left(#1\right)}
\newcommand\gln[2]{\text{GL}_{#1}\left(#2\right)}
\newcommand\glnK[1]{\text{GL}_{#1}\left(\K\right)}
\newcommand\glnR[1]{\text{GL}_{#1}\left(\R\right)}
\newcommand\glnC[1]{\text{GL}_{#1}\left(\C\right)}
\newcommand\dual[1]{{#1}^{*}}

\ExplSyntaxOn
\DeclareExpandableDocumentCommand{\fpdim}{m}
{
  \fp_to_dim:n { round(#1,5) }
}
\ExplSyntaxOff

% Fix \left and \right spacing
\let\originalleft\left
\let\originalright\right
\renewcommand{\left}{\mathopen{}\mathclose\bgroup\originalleft}
\renewcommand{\right}{\aftergroup\egroup\originalright}

\NewDocumentCommand\intervalle{s m m m m m}{
  \IfBooleanTF{#1}{%
    \mathopen{}\left%
      #2{}#3\mathclose{}\mathpunct{}#6#4\mathclose{}\right#5
  }{%
    \delimitershortfall=0pt
    \delimiterfactor=1000
    \mathopen{}%
    \left#2{}#3\mathclose{}\mathpunct{}#6#4\mathclose{}\right#5
  }%
}

\NewDocumentCommand\interff{s m m}{
  \IfBooleanTF{#1}{%
    \intervalle{[}{#2}{#3}{]}{,}
  }{%
    \intervalle*{[}{#2}{#3}{]}{,}
  }%
}

\NewDocumentCommand\interfo{s m m}{
  \IfBooleanTF{#1}{%
    \intervalle{[}{#2}{#3}{[}{,}
  }{%
    \intervalle*{[}{#2}{#3}{[}{,}
  }%
}

\NewDocumentCommand\interof{s m m}{
  \IfBooleanTF{#1}{%
    \intervalle{]}{#2}{#3}{]}{,}
  }{%
    \intervalle*{]}{#2}{#3}{]}{,}
  }%
}

\NewDocumentCommand\interoo{s m m}{
  \IfBooleanTF{#1}{%
    \intervalle{]}{#2}{#3}{[}{,}
  }{%
    \intervalle*{]}{#2}{#3}{[}{,}
  }%
}

\NewDocumentCommand\interffsc{s m m}{
  \IfBooleanTF{#1}{%
    \intervalle{[}{#2}{#3}{]}{;}
  }{%
    \intervalle*{[}{#2}{#3}{]}{;}
  }%
}

\NewDocumentCommand\interfosc{s m m}{
  \IfBooleanTF{#1}{%
    \intervalle{[}{#2}{#3}{[}{;}
  }{%
    \intervalle*{[}{#2}{#3}{[}{;}
  }%
}

\NewDocumentCommand\interofsc{s m m}{
  \IfBooleanTF{#1}{%
    \intervalle{]}{#2}{#3}{]}{;}
  }{%
    \intervalle*{]}{#2}{#3}{]}{;}
  }%
}

\NewDocumentCommand\interoosc{s m m}{
  \IfBooleanTF{#1}{%
    \intervalle{]}{#2}{#3}{[}{;}
  }{%
    \intervalle*{]}{#2}{#3}{[}{;}
  }%
}

\NewDocumentCommand\iinterff{s m m}{
  \IfBooleanTF{#1}{%
    \intervalle{\llbracket}{#2}{#3}{\rrbracket}{,}
  }{%
    \intervalle*{\llbracket}{#2}{#3}{\rrbracket}{,}
  }%
}

\NewDocumentCommand\iinterfo{s m m}{
  \IfBooleanTF{#1}{%
    \intervalle{\llbracket}{#2}{#3}{\llbracket}{,}
  }{%
    \intervalle*{\llbracket}{#2}{#3}{\llbracket}{,}
  }%
}

\NewDocumentCommand\iinterof{s m m}{
  \IfBooleanTF{#1}{%
    \intervalle{\rrbracket}{#2}{#3}{\rrbracket}{,}
  }{%
    \intervalle*{\rrbracket}{#2}{#3}{\rrbracket}{,}
  }%
}

\NewDocumentCommand\iinteroo{s m m}{
  \IfBooleanTF{#1}{%
    \intervalle{\rrbracket}{#2}{#3}{\llbracket}{,}
  }{%
    \intervalle*{\rrbracket}{#2}{#3}{\llbracket}{,}
  }%
}

\NewDocumentCommand\iinterffsc{s m m}{
  \IfBooleanTF{#1}{%
    \intervalle{\llbracket}{#2}{#3}{\rrbracket}{;}
  }{%
    \intervalle*{\llbracket}{#2}{#3}{\rrbracket}{;}
  }%
}

\NewDocumentCommand\iinterfosc{s m m}{
  \IfBooleanTF{#1}{%
    \intervalle{\llbracket}{#2}{#3}{\llbracket}{;}
  }{%
    \intervalle*{\llbracket}{#2}{#3}{\llbracket}{;}
  }%
}

\NewDocumentCommand\iinterofsc{s m m}{
  \IfBooleanTF{#1}{%
    \intervalle{\rrbracket}{#2}{#3}{\rrbracket}{;}
  }{%
    \intervalle*{\rrbracket}{#2}{#3}{\rrbracket}{;}
  }%
}

\NewDocumentCommand\iinteroosc{s m m}{
  \IfBooleanTF{#1}{%
    \intervalle{\rrbracket}{#2}{#3}{\llbracket}{;}
  }{%
    \intervalle*{\rrbracket}{#2}{#3}{\llbracket}{;}
  }%
}

\newcommand*{\segment}[2]{\intervalle{[}{#1}{#2}{]}{}}
\newcommand*{\semicolon}{\mathbin{;}}
\newcommand*{\ptnoname}[2]{\left(#1\mathbin{,}#2\right)}
\newcommand*{\pt}[3]{#1\ptnoname{#2}{#3}}

\newenvironment{nscenter}[1][0.3\topsep]
{\setlength{\topsep}{#1}\par\kern\topsep\centering}
{\par\kern\topsep}

\newcommand{\divides}{\mid}
\newcommand{\notdivides}{\nmid}

\allowdisplaybreaks
\newcommand\conj[1]{\overline{#1}}
\newcommand\coord[2][\mathcal{B}]{\textup{Coord}_{#1}\left(#2\right)}

\newcommand\diagramoneapp[5][1.3]{
  \begin{tikzpicture}[baseline={(0,-0.45ex)},
    every node/.style={outer ysep=0.5, inner ysep=0.5}]%
    \node (n3) at (0,0) {#2};%
    \node (n2) at ($ (n3.east) + (#1,0) $) {#3};%
    \draw[->] (n3) -- (n2)%
    node[midway, above] {#4}%
    node[midway, below]%
    {#5};%
  \end{tikzpicture}%
}

\newcommand\diagramtwoapp[8][1.3]{
  \begin{tikzpicture}[baseline={(0,-0.45ex)},
    every node/.style={outer ysep=0.5, inner ysep=0.5}]%
    \node (n3) at (0,0) {#2};%
    \node (n2) at ($ (n3.east) + (#1,0) $) {#3};%
    \node (n1) at ($ (n2.east) + (#1,0) $) {#4};%
    \draw[->] (n3) -- (n2)%
    node[midway, above] {#5}%
    node[midway, below]%
    {#6};%
    \draw[->] (n2) -- (n1)%
    node[midway, above] {#7}%
    node[midway, below]%
    {#8};%
  \end{tikzpicture}%
}
\newcommand\diagramtwoappandcompo[8][2]{
  \def\az{#1}
  \def\na{#2}
  \def\nb{#3}
  \def\nc{#4}
  \def\apa{#5}
  \def\mata{#6}
  \def\apb{#7}
  \def\matb{#8}
  \diagramtwoappandcompocontinued
}
\newcommand\diagramtwoappandcompocontinued[3][-1.2]{
  \begin{tikzpicture}[baseline={(0,-0.45ex)}]%
    \node (n3) at (0,0) {\na};%
    \node (n2) at ($ (n3.east) + (\az,0) $) {\nb};%
    \node (n1) at ($ (n2.east) + (\az,0) $) {\nc};%
    \draw[->] (n3) -- (n2)%
    node[midway, above] {\apa}%
    node[midway, below]%
    {\mata};%
    \draw[->] (n2) -- (n1)%
    node[midway, above] {\apb}%
    node[midway, below]%
    {\matb};%
    \coordinate (bn2) at ($ (n2) + (0,#1) $);
    \draw[rounded corners, ->] (n3) |- (bn2) -| (n1);
    \draw (bn2)
    node[above] {#2}
    node[below] {#3};
  \end{tikzpicture}%
}


\robustify\inlinelimit
\robustify\inlineleftlimit
\robustify\inlinerightlimit
\robustify\inlinedifflimit
\robustify\inlineseqlimit
\robustify\transposeRight
\robustify\transposeLeft
\robustify\transpose
\robustify\MmnK
\robustify\MnK
\robustify\MmnR
\robustify\MnR
\robustify\MmnC
\robustify\MnC
\robustify\MmnQ
\robustify\MnQ
\robustify\Mat
\robustify\diag
\robustify\MatPass
\robustify\MatFamille
\robustify\Colonne
\robustify\Ligne
\robustify\xrightarrow
\robustify\lineappset
\robustify\lineendo
\robustify\permset
\robustify\opltransv
\robustify\opctransv
\robustify\opldilat
\robustify\opcdilat
\robustify\opligne
\robustify\opcol
\robustify\opcolmat
\robustify\oplex
\robustify\opcex

\renewcommand{\parallel}{\mathbin{\!/\mkern-5mu/\!}}

\newcommand\clefprim[1]{\smash{\underline{#1}}}

\AtBeginDocument{%
  \renewcommand\C{\mathbf{C}}
  \robustify\C
  \renewcommand\angle[2]{\left(#1,#2\right)}
  \renewcommand\d[1]{\text{d}}
  \robustify\angle
  \renewcommand\leq{\leqslant}
  \renewcommand\geq{\geqslant}
  \renewcommand{\Re}{\operatorname{Re}}%
  \renewcommand{\Im}{\operatorname{Im}}%
  \robustify\Re
  \robustify\Im
  \let\temp\phi
  \let\phi\varphi
  \let\varphi\temp
  \let\th\relax  
  \DeclareMathOperator{\th}{th}
  \robustify\th
  \robustify\cnset
  \robustify\class
  \robustify\fullfunction
  \robustify\fullfunctioninline
  \robustify\Vect
  \robustify\im
  \robustify\coord
  \robustify\substack
  \robustify\rang
  \robustify\ln
  \robustify\substack
  \robustify\conj
  \robustify\vvect
  \robustify\left
  \robustify\right
  \robustify\inlinefunction
  \robustify\inlinemapsto
  \robustify\inlinemapstonamed
}
\usepackage{diagbox}

%% MP
\newcommand\charpoly[1]{\chi_{#1}}
\newcommand\spec[1]{\textrm{Sp}(#1)}

% Not in mathjax yet
\newcommand\generatedsubgroup[1]{<#1>} \DeclareMathOperator{\ord}{ord}
\newcommand\ZnZ[1]{\Z/#1\Z}
\newcommand\derp[2]{\frac{\partial{}\,#1}{\partial{}#2}}
\newcommand\derpn[3]{\frac{\partial{\,#1}^{#3}}{\partial{#2}^{#3}}}
\newcommand\der[2]{\frac{\text{d}#1}{\text{d}#2}}
\newcommand\derlong[2]{\frac{\text{d}}{\text{d}#2}\left(#1\right)}
\newcommand\gradient[1]{\nabla{#1}}

\newcommand\cercle[2]{C(#1;#2)}
\newcommand\closeddisc[2]{\overline{D}(#1;#2)}
\newcommand\openeddisc[2]{D(#1;#2)}
\newcommand\closedball[2]{\overline{B}(#1;#2)}
\newcommand\openedball[2]{B(#1;#2)}


%%% Local Variables:
%%% mode: latex
%%% TeX-master: t
%%% End:

\usepackage{pgfornament}
\usepackage{amsthm}
\usepackage[newfloat]{minted}
\tcbuselibrary{minted}

\ExplSyntaxOn
\cs_generate_variant:Nn \regex_extract_once:nnNTF {VVNTF}
\cs_generate_variant:Nn \tl_upper_case:n {V}
\tl_new:N \g_adam_spaced_job_name_tl
\tl_new:N \g_adam_spaced_jobtype
\tl_new:N \g_adam_spaced_jobclasse
\tl_new:N \g_adam_spaced_jobnbr
\tl_new:N \g_adam_spaced_jobdate
\tl_new:N \g_adam_spaced_joby
\tl_new:N \g_adam_spaced_jobm
\tl_new:N \g_adam_spaced_jobd

\tl_new:N \g_adam_spaced_jobclasse_upper
\tl_new:N \g_adam_spaced_jobdescr

\seq_new:N \g_adam_job_name_seq
\seq_new:N \g_adam_job__seq
% rescan \jobname
\tl_gset_rescan:Nnx \g_adam_spaced_job_name_tl { } { \c_sys_jobname_str }
\seq_gset_split:NnV \g_adam_job_name_seq { _ } \g_adam_spaced_job_name_tl
% replace hyphens by spaces
\tl_greplace_all:Nnn \g_adam_spaced_job_name_tl { _ } { ~ }
% \g_adam_job_name_seq
\seq_gpop_left:NN \g_adam_job_name_seq \g_adam_spaced_jobtype
\seq_gpop_left:NN \g_adam_job_name_seq \g_adam_spaced_jobclasse
\seq_gpop_left:NN \g_adam_job_name_seq \g_adam_spaced_jobnbr
\seq_gpop_left:NN \g_adam_job_name_seq \g_adam_spaced_jobdate

\seq_gset_split:NnV \g_adam_job__seq { - } \g_adam_spaced_jobdate
\seq_gpop_left:NN \g_adam_job__seq \g_adam_spaced_joby
\seq_gpop_left:NN \g_adam_job__seq \g_adam_spaced_jobm
\seq_gpop_left:NN \g_adam_job__seq \g_adam_spaced_jobd


\tl_gset:Nx \g_adam_spaced_jobclasse_upper {
  \tl_upper_case:V {\g_adam_spaced_jobclasse}
  }
\tl_gset:Nn \g_adam_spaced_jobdescr {
  \seq_use:Nn \g_adam_job_name_seq {~}
}

\NewDocumentCommand{\doctitle}{}
 {
  % just print the (spaced) file name
  \tl_use:N \g_adam_spaced_jobdescr
 }
\NewDocumentCommand{\doctype}{}
 {
  \tl_to_str:N \g_adam_spaced_jobtype
 }
\NewDocumentCommand{\docclasse}{}
 {
  \tl_to_str:N \g_adam_spaced_jobclasse
}
\NewDocumentCommand{\docnbr}{}
 {
  \tl_to_str:N \g_adam_spaced_jobnbr
 }
\NewDocumentCommand{\docyear}{}
 { \tl_to_str:N \g_adam_spaced_joby}
\NewDocumentCommand{\docmonth}{}
 { \tl_to_str:N \g_adam_spaced_jobm}
\NewDocumentCommand{\docday}{}
 { \tl_to_str:N \g_adam_spaced_jobd}

\NewDocumentCommand{\docclasseupper}{}
 {
  \tl_to_str:N \g_adam_spaced_jobclasse_upper
 }

\NewDocumentCommand{\spacedfilename}{}
 {
  \tl_use:N \g_adam_spaced_job_name_tl
 }
\ExplSyntaxOff


\newcommand\tcbsolution[2]{%
  \IfFileExists{#2}{\pagebreak[3]\tcbsolutionb{#1}{#2}\pagebreak[3]}{}%
}

\NewTotalTColorBox{\tcbsolutionb}{mm}{%
  blank,
  coltitle=ocre,
  breakable,
  fonttitle=\bfseries,
  title={Exercise~\ref{ex-#1}:},
  phantomlabel={solution@#1},
  attach title to upper=\par,
}{\input{#2}}

\newtcblisting{pythoncode}[2][]{%
  before skip=1em,
  after skip=1em,
  title={\ifx\relax#2\relax Code\else #2\fi},
  halign title=center,
  boxsep=1mm,
  left=0mm,
  right=0mm,
  breakable,
  listing engine=minted,
  listing only,
  minted language={python},
  minted options={breaklines, autogobble},
  #1
}

\everymath{\displaystyle}

\fancypagestyle{DS}{
  \setdate{\docyear}{\docmonth}{\docday}
  \fancyhf{}
  \renewcommand{\headrulewidth}{0.4pt}
  \fancyhead[L]{\docclasseupper{} --- \etablissement}
  \fancyhead[R]{Épreuve blanche du \MakeLowercase{\datedayname}~\datedate}
  \fancyfoot[R]{\thepage/%
    \the\numexpr%
    \getpagerefnumber{corfirstpage}-1\relax
  }
  \fancyfoot[L]{Épreuve blanche~\no\docnbr}
}
\pagestyle{DS}
\fancypagestyle{DSfirstpage}{
  \setdate{\docyear}{\docmonth}{\docday}
  \fancyhf{}
  \renewcommand{\headrulewidth}{0.4pt}
  \fancyhead[L]{\docclasseupper{} --- \etablissement}
  \fancyhead[R]{Épreuve blanche du \MakeLowercase{\datedayname}~\datedate}
  \fancyfoot[R]{\thepage/%
    \the\numexpr%
    \getpagerefnumber{corfirstpage}-1\relax
  }
  \fancyfoot[L]{Épreuve blanche~\no\docnbr}
}
\thispagestyle{DSfirstpage}

\newcounter{corpage}
\fancypagestyle{excor}{
  \setdate{\docyear}{\docmonth}{\docday}
  \fancyhf{}
  \renewcommand{\headrulewidth}{0.4pt}

  \fancyhead[L]{\docclasseupper{} --- \etablissement}
  \fancyhead[C]{\doctitle}
  \fancyfoot[R]{%
    \the\numexpr%
    \thepage-\getpagerefnumber{corfirstpage}+1\relax%
    /%
    \the\numexpr%
    \getpagerefnumber{TrueLastPage}-\getpagerefnumber{corfirstpage}+1
    \relax%
  }
  \fancyfoot[L]{Correction de l'épreuve blanche~\no\docnbr}
}


\newcommand{\upperRomannumeral}[1]{\uppercase\expandafter{\romannumeral#1}}
\AfterEndPreamble{
  \pagenumbering{arabic}
  \tcbstartrecording%
  \everymath{\displaystyle}
  \begin{center}
    \begin{tikzpicture}[
      every node/.style={inner sep=10pt}]
      \node[align=center] (vecbox) {
        \LARGE{MATHÉMATIQUES}\\
        Épreuve blanche~\no\docnbr{}\\
        Durée:~\ifcsname dureeds\endcsname\dureeds\else{}4h\fi%
      };%
      \path  (vecbox.south west) to [ornament=88,
      options/.append style={yshift=1pt}] (vecbox.south east);
      \path  (vecbox.north west) to [ornament=88,
      options/.append style={yshift=1pt}] (vecbox.north east);
    \end{tikzpicture}
  \end{center}
  \bigskip{}
  \begin{center}
    \begin{minipage}{0.9\linewidth}
      \itshape%
      Les calculatrices ne sont pas autorisées. Un grand soin devra
      être apporté à la rédaction et à la présentation. Si vous
      constatez ce qui vous semble être une erreur d'énoncé,
      signalez-le et poursuivez votre composition en expliquant les
      raisons des initiatives que vous serez amenés à prendre.
      Rappelons que toute tentative de recherche, même infructueuse,
      sera prise en considération. Enfin les exercices sont
      indépendants et peuvent être traités dans un ordre quelconque.
    \end{minipage}
  \end{center}
  \ifx\relax\consigneds\relax
  \else
  \begin{center}
    \begin{minipage}{0.9\linewidth}
      \itshape{}%
      \consigneds
    \end{minipage}
  \end{center}
  \fi
  \bigskip{}
}

\makeatletter{}
\def\tcbverbatimwrite#1{%
  \@bsphack
  \immediate\openout\tcb@out #1
  \edef\beginputlineno{\the\inputlineno}
  \tcb@verbatim@begin@hook%
  \let\do\@makeother\dospecials
  \catcode`\^^M\active \catcode`\^^I=12
  \def\verbatim@processline{%
    \immediate\write\tcb@out
      {\the\verbatim@line}}%
  \verbatim@start}%'

\def\endtcbverbatimwrite{%
  \tcb@verbatim@end@hook%
  \def\@@percentchar{\@percentchar\@percentchar\@percentchar}
  \immediate\write\tcb@out
    {%
   }
  \immediate\closeout\tcb@out
  \@esphack%
}
\makeatother{}

\newtcbtheorem{exercice}{Exercice}{%
  description delimiters={\og}{\fg},
  theorem style=plain,
  lowerbox=ignored,
  savelowerto=Solutions/\jobname-sol-\thetcbcounter.tex,
  record={\string\tcbsolution{%
      \thetcbcounter%
    }{%
      Solutions/\jobname-sol-\thetcbcounter.tex%
    }
  },
  code={\tcbset{label={ex-\thetcbcounter}}},%
  detach title,
  terminator sign={:~},
  separator sign none,
  breakable, blank, new/crefname={exercice}{exercices},
  new/Crefname={Exercice}{Exercices}, description
  delimiters={\og}{\fg}, fonttitle=\bfseries,coltitle=ocre, enhanced,%
  before skip=1em, after skip=1em, left=0pt, right=0pt,
  fonttitle=\bfseries%
}{ex}

\AtEndDocument{%
  \tcbstoprecording%
  \label{subjectlastpage}%
  \clearpage\label{corfirstpage}%
  \ifhandout%
  \else%
  \ifwacom%
  \else%
  \begin{center}
    \begin{tikzpicture}[
      every node/.style={inner sep=10pt}]
      \node[align=center] (vecbox) {
        \LARGE{MATHÉMATIQUES}\\
        Correction de l'épreuve blanche~\no\docnbr{}
      };%
      \path  (vecbox.south west) to [ornament=88,
      options/.append style={yshift=1pt}] (vecbox.south east);
      \path  (vecbox.north west) to [ornament=88,
      options/.append style={yshift=1pt}] (vecbox.north east);
    \end{tikzpicture}
  \end{center}
  \bigskip{}
  \tcbinputrecords%
  \fi
  \fi
  \label{TrueLastPage}%
}
