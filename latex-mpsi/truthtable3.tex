% \RequirePackage{etoolbox}
% \RequirePackage{expl3}
% \RequirePackage{xparse}
% \RequirePackage{tabularx}

\makeatletter{}
% Boolean operator and checking functions
\def\truthtbl@raiseboolerror#1{%
  \PackageError{Truth table}{'#1' is not a boolean}{}}
\def\truthtbl@checkifbool#1{%
  \if#1V%
  \else\if#1F%
  \else%
  \truthtbl@raiseboolerror{#1}\fi\fi}
\def\tnot#1{%
  \truthtbl@checkifbool{#1}%  
  \if#1V%
  F\else%
  V\fi%
}
\def\tand#1#2{%
  \truthtbl@checkifbool{#1}%
  \truthtbl@checkifbool{#2}%
  \if#1V%
  \if#2V%
  V\else%
  F%
  \fi%
  \else%
  F%
  \fi}
\def\tor#1#2{%
  \truthtbl@checkifbool{#1}%
  \truthtbl@checkifbool{#2}%
  \if#1V%
  V%
  \else%
  \if#2V%
  V\else%
  F%
  \fi\fi}
\def\txor#1#2{%
  \truthtbl@checkifbool{#1}%
  \truthtbl@checkifbool{#2}%
  \if#1V%
  \if#2V%
  F\else%
  V%
  \fi%
  \else%
  \if#2V%
  V\else%
  F%
  \fi\fi}
\def\tnand#1#2{%
  \tnot{\tand{#1}{#2}}
}
\def\timplies#1#2{%
  \tor{\tnot{#1}}{#2}
}
\def\tequiv#1#2{%
  \or{%
    \tand{#1}{#2}%
  }{%
    \tand{\tnot{#1}}{\tnot{#2}}%
  }}

\ExplSyntaxOn
% Convert commas to ampersands
\def\commatoamp#1{\if#1,&\else #1\fi}
\NewDocumentCommand{\replcommaswithamps}{m}{
  \tl_gset:Nx \aux_tl {#1}
  \tl_gset:Nx \g_res_tl {
    \tl_map_function:NN \aux_tl \commatoamp
  }
  \tl_use:N \g_res_tl
}

% Count items in the argument
\NewDocumentCommand \countItems { m } {
  \clist_count:N #1
}
% Count items in the argument
\NewDocumentCommand \countInlineItems { m } {
  \clist_count:n {#1}
}
\ExplSyntaxOff

\def\thruthtbl@setupsomekeys#1#2#3{%
  \ifcase#1\relax%
  \pgfkeysdefargs{/truthtbl/evalexpr #3}{}{#2}%
  \or%
  \pgfkeysdefargs{/truthtbl/evalexpr #3}{##1}{#2}%
  \or%
  \pgfkeysdefargs{/truthtbl/evalexpr #3}{##1,##2}{#2}%
  \or%
  \pgfkeysdefargs{/truthtbl/evalexpr #3}{##1,##2,##3}{#2}%
  \or%
  \pgfkeysdefargs{/truthtbl/evalexpr #3}{##1,##2,##3,##4}{#2}%
  \or%
  \pgfkeysdefargs{/truthtbl/evalexpr #3}{##1,##2,##3,##4,##5}{#2}%
  \or%
  \pgfkeysdefargs{/truthtbl/evalexpr #3}{##1,##2,##3,##4,##5,##6}{#2}%
  \or%
  \pgfkeysdefargs{/truthtbl/evalexpr #3}{##1,##2,##3,##4,##5,##6,##7}{#2}%
  \or%
  \pgfkeysdefargs{/truthtbl/evalexpr #3}{##1,##2,##3,##4,##5,##6,##7,##8}{#2}%
  \or%
  \pgfkeysdefargs{/truthtbl/evalexpr #3}{%
    ##1,##2,##3,##4,##5,##6,##7,##8,##9}{#2}%
  \else%
  \pgfkeys@error{Expected <= 9 arguments, got #2}%
  \fi%

}

\pgfkeysdefnargs{/truthtbl/expr}{2}{%
  \def\truthtbl@exprnbrargs{\countInlineItems{#1}}
  \def\truthtbl@exprargs{#1}
  \pgfkeyssetvalue{/truthtbl/exprargs}{#1}
  \pgfkeyssetvalue{/truthtbl/exprnbrargs}{\countInlineItems{#1}}
  \pgfkeyssetvalue{/truthtbl/exprargs}{#1}
  \newcounter{expri}
  \setcounter{expri}{0}
  \def\doit##1{%
    \stepcounter{expri}%
    \thruthtbl@setupsomekeys{\truthtbl@exprnbrargs}{##1}{\theexpri}}
  \forcsvlist{\doit}{#2}
  \xdef\truthtbl@exprlistnbrelmt{\theexpri}
}

\newif\ifheaderinmathmode
\newif\ifbodyinmathmode
\pgfqkeys{/truthtbl}{%
  expression name/.store in=\truthtbl@exprname,
  expression name/.default={Expr.},
  header in math mode/.is if=headerinmathmode,
  header in math mode/.is if=true,
  body in math mode/.is if=bodyinmathmode,
  body in math mode/.is if=true,
}


% Put all comma-separated strings of #1 boolean in the list
% \thruthtbl@allvalues
\def\truthtbl@putalltruthvaluesoflength#1{%
  \truthtbl@putalltruthvaluesoflengthwithprefix{#1}{}
}

\def\truthtbl@putalltruthvaluesoflengthwithprefix#1#2{%
  \ifnumcomp{#1}{>}{0}{%
    \ifx\relax#2\relax
    \truthtbl@putalltruthvaluesoflengthwithprefix{\numexpr #1-1\relax}{V}%
    \truthtbl@putalltruthvaluesoflengthwithprefix{\numexpr #1-1\relax}{F}%
    \else
    \truthtbl@putalltruthvaluesoflengthwithprefix{\numexpr #1-1\relax}{#2,V}%
    \truthtbl@putalltruthvaluesoflengthwithprefix{\numexpr #1-1\relax}{#2,F}%
    \fi
  }{
    \listgadd{\thruthtbl@allvalues}{#2}
  }
}

\def\drawthruthtable#1{%
  \gdef\thruthtbl@allvalues{}
  \pgfqkeys{/truthtbl}{#1}
  % Put all csvstring in the list \thruthtbl@allvalues
  \truthtbl@putalltruthvaluesoflength{\truthtbl@exprnbrargs}
  % Create the header string of the form ccc...c|c
  \newcount\argind
  \argind=1
  \def\truthtbl@header{}
  \loop
  \advance \argind by 1
  \edef\truthtbl@header{c\truthtbl@header}
  \unless\ifnum \argind>\truthtbl@exprnbrargs
  \repeat
  \edef\truthtbl@header{\truthtbl@header|cc}
  % Create each line.
  \def\headerline{%
    \replcommaswithamps{\truthtbl@exprargs} & \truthtbl@exprname & OKOK\\ \hline}
  \newcounter{currexprind}
  % create a dummy list with
  % same length than the expression list...
  \xdef\truthtable@dummycsv{a}
  \argind=1
  \loop\ifnum \argind<\truthtbl@exprlistnbrelmt
  \advance \argind by 1
  \g@addto@macro\truthtable@dummycsv{,a}
  \repeat
  \xdef\truthtable@alllines{}
  %
  \def\adda##1{(##1);}
  \forlistloop{\adda}{\thruthtbl@allvalues}
  %
  % Here is the real work
  \forlistloop{\dooneline}{\thruthtbl@allvalues}
  \expandafter\tabular\expandafter{\truthtbl@header}
  \headerline
  \truthtable@alllines
  \endtabular
}

\gdef\@addbackslashbutlast#1{%
  #1\gdef\@addbackslashbutlast##1{ \\ ##1}}

% \gdef\finishline#1{
%   \stepcounter{currexprind}
%   \g@addto@macro\truthtable@lineend{%
%     &%
%     \pgfqkeys{/truthtbl}{%
%       evalexpr \thecurrexprind={\truthtable@currargs}}%
%   }%
% }

\gdef\finishline#1{
  \g@addto@macro\truthtable@lineend{%
    & \pgfqkeys{/truthtbl}{%
      evalexpr 1={V,F}}%
  }%
  \g@addto@macro\truthtable@lineend{%
    & \pgfqkeys{/truthtbl}{%
      evalexpr 2={V,F}}%
  }%
  \show\g@addto@macro
}

\gdef\dooneline#1{%
  \setcounter{currexprind}{0}
  \xdef\truthtable@lineend{}
  \xdef\truthtable@currargs{#1}
  \g@addto@macro\truthtable@alllines{%
      \replcommaswithamps{#1}%
      \forlistloop{\finishline}{\truthtable@dummycsv}
      \truthtable@lineend \\
    }
}


% Hopefully usefull example
% \drawthruthtable{%
%   expression name={\(\neg{}P\wedge{}Q\)},
%   expr={P,Q}{\strut\txor{#1}{#2}}
% }
\makeatother{}
